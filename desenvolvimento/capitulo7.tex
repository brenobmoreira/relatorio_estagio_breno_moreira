\chapter{Capítulo 7}

Conclusões gerais: "resumão" do que foi feito e dos resultados globais, limitações do que foi desenvolvido e pressupostos assumidos, e sugestão para trabalhos futuros de continuação.

Síntese pessoal, objetiva, sucinta e interpretada dos resultados do trabalho.
Grosso modo, deve-se apresentar um resumo do que foi feito, dos resultados globais (frente aos objetivos inicialmente traçados). Exemplos:

\begin{itemize}
\item o método deu certo? funcionou? deu o resultado esperado? Foi melhor que o método anterior?
\item impactos organizacionais, tecnológicos, financeiros, éticos, ecológicos, etc., tidos (ou potencialmente a ter) com a introdução do que foi proposto.
\end{itemize}


De forma complementar, se pertinente, sugestões para trabalhos futuros de continuação.