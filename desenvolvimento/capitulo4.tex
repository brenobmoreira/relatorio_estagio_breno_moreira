\chapter{Requisitos do projeto}

Este capítulo descreve os requisitos do projeto definidos para a solução do pipeline. Os requisitos são classificados com gerais, funcionais ou não funcionais e representam as diretrizes que guiaram o desenvolvimento do sistema. O foco da implementação durante o periodo do estágio foi atender a este conjunto de requisitos.

\section{Requisitos Gerais (RG)}

\begin{itemize}
  \item \textbf{RG1: Modularidade}

  A arquitetura do sistema deve ser modular, permitindo que diferentes pipelines sejam desenvolvidos, testados e implementados de forma indepentende.

  \item \textbf{RG2: Configurabilidade}
  
  Os pipelines devem ser definidos de forma declarativa através dos arquivos de configurações (YAML e JSON).

  \item \textbf{RG3: Integração com Nuvem}
  
  A solução deve utilizar o Azure Blob Storage como repositório para dados brutos e processados.
\end{itemize}


\section{Requisitos Funcionais (RF)}

\begin{itemize}
  \item \textbf{RF1: Monitoramento automatizado de fontes}

  O sistema deve ser capaz de monitorar fontes de dados externas de forma automática, detectando arquivos novos ou atualizados baseados na última checagem.

  \item \textbf{RF2: Ingestão automatizada para data warehouse}
  
  Quando detectar um novo arquivo, o sistema deve ser capaz de realizar o download e carregar os dados brutos para o Azure.

  \item \textbf{RF3: Orquestração baseada em mapeamento}
  
  Deve ser utilizado um arquivo de mapeamento para a orquestração dos pipelines, que identifiquem qual pipeline processar baseado no arquivo atualizado e disparar sua execução.

  \item \textbf{RF4: Leitura de diferentes formatos}
  
  O pipeline deve ser capaz de ler os arquivos brutos em diferentes formatos diretamente do Azure. (\.csv, \.xlsx, \.parquet, \.txt)

  \item \textbf{RF5: Execução de SQL}
  
  O sistema deve ser capaz de executar os scripts em SQL parar poder realizar o tratamento dos dados.

  \item \textbf{RF6: Carga de arquivos processados}
  
  Ao final do pipeline, o sistema deve ser capaz de transformar os dados em arquivos processados e salvos no Azure.

\end{itemize}

\section{Requisitos Não Funcionais (RNF)}

\begin{itemize}
    \item \textbf{RNF1: Perfomance}
    
    O processamento das transformações deve ser eficiente e capaz de processar grandes volumes de dados de forma rápida.

    \item \textbf{RNF2: Escalabilidade}
    
    A arquitetura baseada em nuvem deve permitir que o sistema escale horizontalmente para acomodar o crescimento do volume de dados e o número de pipelines.

    \item \textbf{RNF3: Manutenibilidade}
    
    O sistema deve ser de fácil manutenção. A configuração via arquivos JSON e YAML é essencial para o sistema.

    \item \textbf{RNF4: Automação}
    
    O sistema deve ser ser automatizado de ponta a ponta. Da detecção de novos dados até a carga dos arquivos processados.

\end{itemize}